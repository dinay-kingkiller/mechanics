\documentclass{amsart}

% Import Packages
\usepackage{amsmath}
\usepackage{amsfonts}
\usepackage{amsthm}

% Declare Theorem Environments
\newtheorem{theorem}{Theorem}[section]
\newtheorem{lemma}[theorem]{Lemma}
\theoremstyle{definition}
\newtheorem{definition}[theorem]{Definition}
\theoremstyle{remark}
\newtheorem*{remark}{Remark}
\newtheorem*{example}{Example}

% Declare Math Operators
\DeclareMathOperator{\arctanh}{arctanh}
\DeclareMathOperator{\diag}{diag}
\DeclareMathOperator{\sech}{sech}
\DeclareMathOperator{\tr}{tr}

\begin{document}
\title{Symmetries, Spacetime, and Relativity}
\author{}
\date{\today}

\begin{abstract}
  These notes embark on an exploration of the geometric and algebraic foundations underpinning special and general relativity.
  Starting with the fundamental concepts of symmetry groups and their associated algebras, we build from basic spatial rotations to the full symmetries of flat spacetime and their implications for particle properties.
  The journey then progresses to the curved spacetime of general relativity, aiming to elucidate key physical phenomena such as the anomalous precession of Mercury's orbit, the relativistic intricacies of satellite timekeeping, and the fascinating physics of rotating black holes.
  The treatment seeks to bridge abstract formalism with physical insight, providing a foundation for readers new to manifolds, tangent space algebras, and the geometric description of gravity.
\end{abstract}

\maketitle

\section{Introduction}
\label{sec:intro}

The goal of these notes is to develop a clear and mathematically grounded understanding of the principles underlying classical mechanics, special relativity, and general relativity.
We aim to build the formalism necessary to explore key physical phenomena, such as the precession of Mercury's orbit, the intricacies of satellite timekeeping, and the nature of spinning black holes.

We begin in Section~\ref{sec:cont_symm} by introducing continuous symmetries and their algebraic structure using the familiar example of spatial rotations and the classical symmetry group of spacetime.
Section~\ref{sec:sr} then transitions to the fabric of flat spacetime and the symmetries of special relativity, exploring the homogeneous isometry group of flat spacetime, its algebra, and the fundamental kinematic consequences.
The full symmetry group of flat spacetime and its role in defining particle properties like mass and spin through its Casimir invariants, is discussed in Section~\ref{sec:poincare_particle_states}.

Following this, Section~\ref{sec:transition_gr} discusses the conceptual leap from special to general relativity, motivating the idea of gravity as spacetime curvature.
The mathematical language of curved spacetime, differential geometry, is developed in Section~\ref{sec:math_gr}.
This leads to Einstein's field equations in Section~\ref{sec:efe}, which describe how matter and energy dictate this curvature.

With these tools, Section~\ref{sec:schwarzschild_tests} examines the spacetime geometry around a non-rotating, uncharged spherical mass, applying this to understand phenomena like the perihelion precession of Mercury, relativistic effects on clocks, and gravitational lensing.
More complex scenarios involving rotating and charged massive objects, including the concept of an ergosphere, are explored in Section~\ref{sec:kerr_etc}.
Finally, Section~\ref{sec:future_phenomena} briefly touches upon further striking phenomena and future directions in the study of spacetime and gravity.

Throughout, the emphasis is on building intuition and formalism simultaneously, focusing on precise mathematical definitions and constructive examples.
The intended audience is readers with some background in calculus and linear algebra, but no prior exposure to group theory or differential geometry is assumed.

\section{Continuous Symmetries and Their Algebraic Structure}
\label{sec:cont_symm}

In physics and mathematics, continuous symmetry groups describe transformations that preserve certain structures.
These groups are smooth spaces with operations that vary smoothly.
The algebra associated with such a group is the tangent space at the identity element, equipped with a bracket operation that encodes the infinitesimal structure of the group.

\subsection{Foundational Example: The Group of Spatial Rotations ($SO(3)$)}
\label{subsec:so3}
A key example is the group of rotations in three dimensions, consisting of all orthogonal $3 \times 3$ matrices with determinant 1.
These rotations preserve lengths and orientations in three-dimensional space.
We begin with the group of rotations in three dimensions and its algebra, which provides an accessible example to understand generators, commutation relations, and matrix exponentiation.
This lays the groundwork for the more complex groups appearing in relativity.

\begin{definition}
  The rotation group is defined as
  \begin{equation*}
    \mathrm{SO}(3) = \left\{ R \in \mathbb{R}^{3 \times 3} \middle|
    R^\top R = I, \quad \det(R) = 1 \right\}.
  \end{equation*}
\end{definition}

\begin{theorem}\label{thm:so3_algebra_skew_symmetric}
  The associated algebra consists of all $3 \times 3$ real skew-symmetric matrices:
  \begin{equation*}
    \mathfrak{so}(3) = \left\{ X \in \mathbb{R}^{3 \times 3} \middle|
    X^\top = -X \right\}.
  \end{equation*}
\end{theorem}

\begin{proof}
  Consider a smooth curve $R(t): \mathbb{R} \to \mathrm{SO}(3)$ such that $R(0) = I$.
  Since $R(t)$ is orthogonal,
  \begin{equation*}
    R(t)^\top R(t) = I.
  \end{equation*}
  Differentiate both sides with respect to $t$:
  \begin{equation*}
    \frac{d}{dt} \left( R(t)^\top R(t) \right) = \frac{d}{dt} I = 0.
  \end{equation*}
  Applying the product rule,
  \begin{equation*}
    \dot{R}(t)^\top R(t) + R(t)^\top \dot{R}(t) = 0.
  \end{equation*}
  Evaluating at $t=0$, where $R(0) = I$,
  \begin{equation*}
    \dot{R}(0)^\top I + I^\top \dot{R}(0) = \dot{R}(0)^\top + \dot{R}(0) = 0.
  \end{equation*}
  Define $X := \dot{R}(0)$.
  The above equation implies
  \begin{equation*}
    X^\top = -X,
  \end{equation*}
  so $X \in \mathfrak{so}(3)$ is skew-symmetric.
\end{proof}

\begin{definition}
  The basis elements of tangent space algebra $\mathfrak{so}(3)$ are
  \begin{equation*}
    J_1 =
    \begin{pmatrix}
      0 & 0 & 0 \\
      0 & 0 & -1 \\
      0 & 1 & 0
    \end{pmatrix}, \quad
    J_2 =
    \begin{pmatrix}
      0 & 0 & 1 \\
      0 & 0 & 0 \\
      -1 & 0 & 0
    \end{pmatrix}, \quad
    J_3 =
    \begin{pmatrix}
      0 & -1 & 0 \\
      1 & 0 & 0 \\
      0 & 0 & 0
    \end{pmatrix}.
  \end{equation*}
\end{definition}

\begin{theorem}\label{thm:so3_commutation_relations}
  These satisfy the commutation relations:
  \begin{equation*}
    [J_i, J_j] = \epsilon_{ijk} J_k,
  \end{equation*}
  where $\epsilon_{ijk}$ is the fully antisymmetric symbol with
  \begin{equation*}
    \epsilon_{ijk} =
    \begin{cases}
      +1 & \text{if } (i,j,k) \text{ is an even permutation of } (1,2,3), \\
      -1 & \text{if } (i,j,k) \text{ is an odd permutation of } (1,2,3), \\
      0 & \text{if any two indices are equal}.
    \end{cases}
  \end{equation*}
\end{theorem}
\begin{proof}
  This can be proven explicitly by multiplying each pair of generators and computing their commutators.
  For example,
  \begin{equation*}
    [J_1, J_2] := J_1 J_2 - J_2 J_1 = J_3,
  \end{equation*}
  and similarly for the other pairs, confirming the relation
  \begin{equation*}
    [J_i, J_j] = \epsilon_{ijk} J_k.
  \end{equation*}
\end{proof}

\begin{theorem}\label{thm:so3_exp_map_in_SO3}
  The image of the exponential map of $\mathfrak{so}(3)$ is contained in $\mathrm{SO}(3)$:
  \begin{equation*}
    \forall X \in \mathfrak{so}(3):\ \exp(X) \in \mathrm{SO}(3).
  \end{equation*}
\end{theorem}
\begin{proof}
  Let $R = \exp(X)$ for some $X \in \mathfrak{so}(3)$.
  Since $\mathfrak{so}(3)$ consists of skew-symmetric matrices, we have $X^\top = -X$.
  Using the identity $\left(\exp(X)\right)^\top = \exp(X^\top)$, it follows:
  \begin{equation*}
    R^\top R = \exp(X^\top)\exp(X) = \exp(-X)\exp(X).
  \end{equation*}
  Since $X$ commutes with $-X$ (as $[X, -X] = 0$ for all matrices), we may combine the exponents:
  \begin{equation*}
    R^\top R = \exp(-X + X) = \exp(0) = I,
  \end{equation*}
  so $R$ is orthogonal.
  Next, we use the identity $\det(\exp(X)) = \exp(\tr(X))$, valid for all square matrices $X$.
  Since every $X \in \mathfrak{so}(3)$ is skew-symmetric, we have $\tr(X) = 0$, so:
  \begin{equation*}
    \det(\exp(X)) = \exp(\tr(X)) = \exp(0) = 1.
  \end{equation*}
  Therefore, $R = \exp(X)$ is a special orthogonal matrix: $\exp(X) \in \mathrm{SO}(3)$.
\end{proof}

\subsection{Broader Example: The Classical Symmetry Group of Spacetime}
\label{subsec:csg}
\begin{definition}
  The \emph{classical symmetry group}, denoted $\mathrm{CSG}(3)$, is the group associated with classical physics.
  It acts on $\mathbb{R}^4$ of the form:
  \begin{equation*}
    (t, \vec{x}) \mapsto (t + \tau, R \vec{x} + \vec{v} t + \vec{a}),
  \end{equation*}
  where $R \in \mathrm{SO}(3)$ is a rotation described in the previous subsection, $\vec{v} \in \mathbb{R}^3$ is a velocity, $\vec{a} \in \mathbb{R}^3$ is a spatial translation, and $\tau \in \mathbb{R}$ is a time translation.
\end{definition}
\begin{definition}
  If we use an augmented spacetime 5-vector in homogeneous coordinates, $(x^0, x^1, x^2, x^3, x^4) \equiv (t,x,y,z,h)$ where $h:=1$ is the homogeneous coordinate, we can define the classical symmetry algebra $\mathfrak{csg}(3)$ in terms of its generators.
  The generators are defined by their non-zero matrix elements $(M)_{rc}$ (row $r$, column $c$, 0-indexed) as follows:
    \begin{itemize}
        \item Rotation generators:
        \begin{itemize}
            \item $J_1$: $(J_1)_{23} = -1, (J_1)_{32} = 1$
            \item $J_2$: $(J_2)_{13} = 1, (J_2)_{31} = -1$
            \item $J_3$: $(J_3)_{12} = -1, (J_3)_{21} = 1$
        \end{itemize}
        \item Boost generators:
        \begin{itemize}
            \item $K_1$: $(K_1)_{01} = 1$
            \item $K_2$: $(K_2)_{02} = 1$
            \item $K_3$: $(K_3)_{03} = 1$
        \end{itemize}
        \item Spatial translation generators:
        \begin{itemize}
            \item $P_1$: $(P_1)_{14} = 1$
            \item $P_2$: $(P_2)_{24} = 1$
            \item $P_3$: $(P_3)_{34} = 1$
        \end{itemize}
        \item Time translation generator:
        \begin{itemize}
            \item $H$: $(H)_{04} = 1$
        \end{itemize}
    \end{itemize}
    All other matrix elements are zero for each generator.
\end{definition}

\begin{theorem}[Commutator relations of the classical symmetry algebra $\mathfrak{csg}(3)$]\label{thm:csg3_commutation_relations}
  The generators \(J_i\) (rotations), \(K_i\) (boosts), \(P_i\) (spatial translations), and \(H\) (time translation) satisfy the following commutation relations:
  \begin{gather*}
    [J_i, J_j] = \epsilon_{ijk} J_k, \quad
    [J_i, K_j] = \epsilon_{ijk} K_k, \quad
    [J_i, P_j] = \epsilon_{ijk} P_k, \\
    [K_i, K_j] = 0, \quad
    [P_i, P_j] = 0, \quad
    [K_i, P_j] = 0, \\
    [H, J_i] = 0, \quad
    [H, P_i] = 0, \quad
    [H, K_i] = P_i,
  \end{gather*}
  where $\epsilon_{ijk}$ is the fully antisymmetric symbol.
\end{theorem}
\begin{proof}
  Similar to the $\mathrm{SO}(3)$ relations found in theorem \ref{thm:so3_commutation_relations}, this can be proven explicitly by multiplying each pair of generators and computing their commutators.
  For example,
  \begin{equation*}
    [J_1, J_2] := J_1 J_2 - J_2 J_1 = J_3,
  \end{equation*}
  The other pairs can be easily checked the same way.
\end{proof}

\section{The Fabric of Flat Spacetime and its Symmetries}
\label{sec:sr}

\subsection{Redefining Space and Time}
\label{subsec:sr_postulates}
The theory of \emph{special relativity} revolutionized our understanding of space and time.
It is built upon two fundamental postulates:

\begin{enumerate}
  \item \textbf{The Principle of Relativity:} The laws of physics take the same form in all inertial frames of reference.
  This means there is no ``absolute'' rest frame; any inertial frame is equally valid for describing physical phenomena.
  \item \textbf{The Principle of Invariant Light Speed:} The speed of light in a vacuum, denoted $c$, has the same value for all inertial observers, regardless of the motion of the light source or the motion of the observer.
\end{enumerate}
These postulates, particularly the second one, have profound consequences, leading to a departure from classical notions of space and time and requiring a new set of transformation laws between inertial frames.
The invariance of the speed of light implies the invariance of the spacetime interval, which is the cornerstone for defining the geometry of flat spacetime and its symmetry group.

\subsection{Symmetries of Flat Spacetime}
\label{subsec:so13_group}
\begin{definition}[Proper orthochronous spacetime symmetry group]
  Let $\eta$ be the flat spacetime metric tensor:
  \begin{equation*}
    \eta = \diag(-1, +1, +1, +1).
  \end{equation*}
  The group of real $4 \times 4$ matrices $\Lambda$ preserving this bilinear form satisfies
  \begin{equation*}
    \Lambda^\top \eta \Lambda = \eta.
  \end{equation*}
  The subgroup of these matrices with determinant $+1$ and with $\Lambda^0{}_0 \geq 1$ (preserving time orientation) is called the \emph{proper orthochronous spacetime symmetry group}, denoted
  \begin{equation*}
    \mathrm{SO}^+(1,3) = \left\{ \Lambda \in \mathrm{GL}(4,\mathbb{R}) \middle|
    \Lambda^\top \eta \Lambda = \eta,\ \det \Lambda = 1,\ \Lambda^0{}_0 \geq 1 \right\}.
  \end{equation*}
\end{definition}

\begin{theorem}\label{thm:so3_rotations_in_so13}
  Rotations in $\mathrm{SO}(3)$ are a subgroup of $\mathrm{SO}^+(1,3)$.
  For $R_3\in\mathrm{SO}(3)$, the $4 \times 4$ matrix
  \begin{equation*}
    \Lambda_R =
    \begin{pmatrix}
      1 & 0 \\
      0 & R_3
    \end{pmatrix} \text{ is in } \mathrm{SO}^+(1,3).
  \end{equation*}
\end{theorem}
\begin{proof}
  We check the conditions for $\Lambda_R \in \mathrm{SO}^+(1,3)$:
  \begin{enumerate}
      \item $\Lambda_R^\top \eta \Lambda_R = \eta$:
        \begin{align*}
          \Lambda_R^\top\eta \Lambda_R &=
          \begin{pmatrix}
            1 & 0 \\
            0 & R_3
          \end{pmatrix}^\top
          \begin{pmatrix}
            -1 & 0 \\
            0 & I_3
          \end{pmatrix}
          \begin{pmatrix}
            1 & 0 \\
            0 & R_3
          \end{pmatrix} \\
          &=
          \begin{pmatrix}
            1 & 0 \\
            0 & R_3^\top
          \end{pmatrix}
          \begin{pmatrix}
            -1 & 0 \\
            0 & I_3
          \end{pmatrix}
          \begin{pmatrix}
            1 & 0 \\
            0 & R_3
          \end{pmatrix} \\
          &=
          \begin{pmatrix}
            1 & 0 \\
            0 & R_3^\top
          \end{pmatrix}
          \begin{pmatrix}
            -1 & 0 \\
            0 & R_3
          \end{pmatrix} \\
          &=
          \begin{pmatrix}
            -1 & 0 \\
            0 & R_3^\top R_3
          \end{pmatrix}.
        \end{align*}
      Since $R_3 \in \mathrm{SO}(3)$, we have $R_3^\top R_3 = I_3$.
      Thus,
      \begin{equation*}
        \Lambda_R^\top\eta \Lambda_R =
        \begin{pmatrix}
          -1 & 0 \\
          0 & I_3
        \end{pmatrix}
        = \eta.
      \end{equation*}
      \item $\det(\Lambda_R) = 1$:
      \begin{equation*}
        \det(\Lambda_R) = \det(1) \cdot \det(R_3) = 1 \cdot 1 = 1,
      \end{equation*}
      since $\det(R_3)=1$ for $R_3 \in \mathrm{SO}(3)$.
      \item $(\Lambda_R)^0{}_0 \geq 1$:
      The $(0,0)$ component of $\Lambda_R$ is $(\Lambda_R)^0{}_0 = 1$, which satisfies the requirement.
  \end{enumerate}
  All conditions are met, so $\Lambda_R \in \mathrm{SO}^+(1,3)$.
\end{proof}

\begin{lemma}\label{lem:boost_x_direction_so11}
  For a boost in the $x$-direction that preserves the interval defined by the flat metric, the transformation matrix $\Lambda\in\mathrm{SO}(1,1)$ (a subgroup of $\mathrm{SO}^+(1,3)$ acting on $t,x$) can be written as
  \begin{equation*}
    \Lambda =
    \begin{pmatrix}
      \gamma & -\gamma v \\
      -\gamma v & \gamma
    \end{pmatrix},
  \end{equation*}
  where $\gamma = \frac{1}{\sqrt{1 - v^2}}$ and the \emph{speed} $v \in (-1,1)$ is expressed in natural units.
  Alternatively, in terms of the \emph{rapidity} parameter $\phi := \arctanh v$, the matrix becomes
  \begin{equation*}
    \Lambda =
    \begin{pmatrix}
      \cosh \phi & -\sinh \phi \\
      -\sinh \phi & \cosh \phi
    \end{pmatrix}.
  \end{equation*}
\end{lemma}
\begin{proof}
  Consider a transformation acting only on the $t$ and $x$ coordinates:
  \begin{equation*}
    \begin{pmatrix}
      t' \\
      x'
    \end{pmatrix} =
    \begin{pmatrix}
      A & B \\
      C & D
    \end{pmatrix}
    \begin{pmatrix}
      t \\
      x
    \end{pmatrix}.
  \end{equation*}
  Let $\Lambda\in\mathrm{SO}(1,1)$
  \begin{equation*}
    \Lambda =
    \begin{pmatrix}
      A & B \\
      C & D
    \end{pmatrix}.
  \end{equation*}
  We require this transformation to preserve the 2D flat spacetime interval, defined by $\eta := \diag(-1, 1)$.
  The condition is $\Lambda^\top \eta \Lambda = \eta$.
  Expanding this:
  \begin{equation*}
    \begin{pmatrix}
      A & C \\
      B & D
    \end{pmatrix}
    \begin{pmatrix}
      -1 & 0 \\
      0 & 1
    \end{pmatrix}
    \begin{pmatrix}
      A & B \\
      C & D
    \end{pmatrix}
    =
    \begin{pmatrix}
      -A^2 + C^2 & -AB + CD \\
      -AB + CD & -B^2 + D^2
    \end{pmatrix}
    =
    \begin{pmatrix}
      -1 & 0 \\ 0 & 1
    \end{pmatrix}.
  \end{equation*}
  This yields the system of equations:
  \begin{align*}
    -A^2 + C^2 &= -1  \\
    -AB + CD &= 0  \\
    -B^2 + D^2 &= 1
  \end{align*}
  And the determinant condition for a special transformation (unit determinant):
  \begin{align*}
    AD-BC &= 1 
  \end{align*}
  From $-A^2 + C^2 = -1$, we rearrange to $A^2 = 1 + C^2$.
  Since $C^2 \geq 0$, this implies $A^2 \geq 1$, so $|A| \geq 1$.
  To ensure that the transformation does not reverse the direction of time (i.e., it is orthochronous), we impose the physical condition $A > 0$.
  Combined with $|A| \geq 1$, this leads to the requirement $A \geq 1$.
  Similarly, from $-B^2 + D^2 = 1$, we rearrange to $D^2 = 1 + B^2$.
  Since $B^2 \geq 0$, this implies $D^2 \geq 1$, so $|D| \geq 1$.
  The choice between $D \geq 1$ and $D \leq -1$ determines the relative orientation of the spatial $x'$-axis.
  For a ``standard'' boost along the positive $x$-direction, we choose the convention $D \geq 1$.
  Given $A \geq 1$, the equation $A^2 - C^2 = 1$ allows us to parameterize $A = \cosh\phi_A$ and $C = s_C \sinh\phi_A$ for some real parameter $\phi_A$ and $s_C = \pm 1$.
  Similarly, given $D \geq 1$, the equation $D^2 - B^2 = 1$ allows us to parameterize $D = \cosh\phi_D$ and $B = s_B \sinh\phi_D$ for some real parameter $\phi_D$ and $s_B = \pm 1$.
  Now substitute these parameterizations into $-AB + CD = 0$ ($AB = CD$):
  \begin{equation*}
    (\cosh\phi_A)(s_B \sinh\phi_D) = (s_C \sinh\phi_A)(\cosh\phi_D).
  \end{equation*}
  Assuming a non-trivial boost (so that $\sinh\phi_A \neq 0$ and $\sinh\phi_D \neq 0$), we can divide by the product $\cosh\phi_A \cosh\phi_D$.
  Since $\cosh\phi_A \ge 1$ and $\cosh\phi_D \ge 1$, this yields:
  \begin{equation*}
    s_B \tanh\phi_D = s_C \tanh\phi_A.
  \end{equation*}
  For a single, simple transformation (represented by a single rapidity $\phi$), we expect $\phi_A$ and $\phi_D$ to be the same or directly related.
  If we set $\phi_A = \phi_D =: \phi$, then this equation implies $s_B = s_C =: s$.
  So our parameters become: $A = \cosh\phi$, $C = s \sinh\phi$, $D = \cosh\phi$, $B = s \sinh\phi$.
  Next, use the determinant condition $AD-BC=1$:
  \begin{equation*}
    (\cosh\phi)(\cosh\phi) - (s \sinh\phi)(s \sinh\phi) = \cosh^2\phi - s^2 \sinh^2\phi = 1.
  \end{equation*}
  Since $s^2 = (\pm 1)^2 = 1$, this simplifies to $\cosh^2\phi - \sinh^2\phi = 1$.
  This is the fundamental hyperbolic identity, which is always true for any real $\phi$.
  Thus, the parameters $A=\cosh\phi, D=\cosh\phi, B=s\sinh\phi, C=s\sinh\phi$ satisfy all conditions derived from metric preservation and unit determinant.
  The matrix is
  \begin{equation*}
    \Lambda =
    \begin{pmatrix}
      \cosh\phi & s\sinh\phi \\
      s\sinh\phi & \cosh\phi
    \end{pmatrix}.
  \end{equation*}
  The specific form stated in the lemma,
  \begin{equation*}
    \Lambda =
    \begin{pmatrix}
      \cosh\phi & -\sinh\phi \\
      -\sinh\phi & \cosh\phi
    \end{pmatrix},
  \end{equation*}
  corresponds to choosing the sign factor $s=-1$.
  This means $A = \cosh\phi, B = -\sinh\phi, C = -\sinh\phi, D = \cosh\phi$.
  This is the standard parameterization for a boost.
  To relate $\phi$ to the speed $v$: consider the origin of the $S'$ frame ($x'=0$).
  It moves with velocity $v$ in the $S$ frame, so its worldline is $x=vt$.
  Substituting into $x' = C t + D x$:
  \begin{equation*}
    0 = C t + D (vt) = (C + Dv)t.
  \end{equation*}
  For this to hold for all $t \neq 0$, we need $C+Dv=0$, so $v = -C/D$.
  Using $C = -\sinh\phi$ and $D = \cosh\phi$:
  \begin{equation*}
    v = -(-\sinh\phi) / (\cosh\phi) = \tanh\phi.
  \end{equation*}
  Thus, the rapidity $\phi = \arctanh v$.
  Given $v = \tanh\phi$:
  \begin{equation*}
    1 - v^2 = 1 - \tanh^2\phi = \sech^2\phi = \frac{1}{\cosh^2\phi}.
  \end{equation*}
  Letting $\gamma:=\frac{1}{\sqrt{1-v^2}}$, $\cosh\phi = \frac{1}{\sqrt{1-v^2}}=\gamma$ and $\sinh\phi = \tanh\phi \cosh\phi = v\gamma$.
  This implies that the speed $|v| < 1$ (since $\phi$ is real, $\tanh\phi \in (-1,1)$).
  Substituting $\cosh\phi = \gamma$ and $\sinh\phi = v\gamma$ back into the matrix form $A=\cosh\phi, B=-\sinh\phi, C=-\sinh\phi, D=\cosh\phi$ gives
  \begin{equation*}
    \Lambda =
    \begin{pmatrix}
      \gamma & -\gamma v \\
      -\gamma v & \gamma
    \end{pmatrix}.
  \end{equation*}
\end{proof}

\begin{theorem}\label{thm:general_boost_so13}
  A general boost in an arbitrary direction $\hat{n}$ (a unit 3-vector) with speed $v$ is given by the $4 \times 4$ matrix $\Lambda_B \in \mathrm{SO}^+(1,3)$:
  \begin{equation*}
    \Lambda_B =
    \begin{pmatrix}
      \gamma & -\gamma v \hat{n}^\top \\
      -\gamma v \hat{n} & I_3 + \left(\gamma-1\right)\hat{n}\hat{n}^\top
    \end{pmatrix}
  \end{equation*}
  where $\gamma=\frac{1}{\sqrt{1-v^2}}$. In terms of rapidity $\phi=\arctanh v$:
  \begin{equation*}
    \Lambda_B =
    \begin{pmatrix}
      \cosh \phi & -\sinh \phi \hat{n}^\top \\
      -\sinh \phi \hat{n} & I_3 + (\cosh \phi - 1) \hat{n} \hat{n}^\top
    \end{pmatrix}.
  \end{equation*}
  % TODO: Proof for general boost (Theorem~\ref{thm:general_boost_so13}) to be detailed or reference provided.
\end{theorem}

\subsection{The Algebra of Spacetime Symmetries ($\mathfrak{so}(1,3)$)}
\label{subsec:so13_algebra}

\begin{definition}
  The algebra associated with the group $\mathrm{SO}^+(1,3)$ is denoted $\mathfrak{so}(1,3)$.
  It consists of all \(4 \times 4\) real matrices \(X\) such that a curve $\Lambda(s) = \exp(sX)$ is in $\mathrm{SO}^+(1,3)$ for small $s$.
  Differentiating $\Lambda(s)^\top \eta \Lambda(s) = \eta$ at $s=0$ (where $\Lambda(0)=I$) yields the condition for elements $X \in \mathfrak{so}(1,3)$:
  \begin{equation*}
    X^\top \eta + \eta X = 0.
  \end{equation*}
  This condition means $X$ is "$\eta$-antisymmetric" or "$\eta$-skew-adjoint".
  A convenient basis for this 6-dimensional algebra, acting on coordinates
  \begin{equation*}
    (x^0, x^1, x^2, x^3) \equiv (t,x,y,z),
  \end{equation*}
  is given by the following generators:
  \begin{itemize}
    \item Rotation generators:
      \begin{itemize}
        \item $J_1$: $(J_1)_{23} = -1, (J_1)_{32} = 1$
        \item $J_2$: $(J_2)_{13} = 1, (J_2)_{31} = -1$
        \item $J_3$: $(J_3)_{12} = -1, (J_3)_{21} = 1$
      \end{itemize}
    \item Boost generators:
      \begin{itemize}
        \item $K_1$: $(K_1)_{01} = -1, (K_1)_{10} = -1$
        \item $K_2$: $(K_2)_{02} = -1, (K_2)_{20} = -1$
        \item $K_3$: $(K_3)_{03} = -1, (K_3)_{30} = -1$
    \end{itemize}
  \end{itemize}
  Note: These generators $X$ indeed satisfy $X^\top \eta + \eta X = 0$ with $\eta=\diag(-1,1,1,1)$.
  The verification for any specific generator involves direct matrix multiplication and addition, confirming the condition.
  % TODO: This should be a theorem
\end{definition}

\begin{theorem}[Commutator relations of the algebra $\mathfrak{so}(1,3)$]\label{thm:so13_algebra_commutation_relations}
  The generators $J_i$ (rotations) and $K_i$ (boosts) satisfy the following commutation relations:
  \begin{align*}
    [J_i, J_j] &= \epsilon_{ijk} J_k, \\
    [J_i, K_j] &= \epsilon_{ijk} K_k, \\
    [K_i, K_j] &= -\epsilon_{ijk} J_k,
  \end{align*}
  where $\epsilon_{ijk}$ is the fully antisymmetric symbol.
\end{theorem}
\begin{proof}
    Similar to the $\mathrm{SO}(3)$ relations found in theorem \ref{thm:so3_commutation_relations}, this can be proven explicitly by multiplying each pair of generators and computing their commutators.
  For example,
  \begin{equation*}
    [J_1, J_2] := J_1 J_2 - J_2 J_1 = J_3,
  \end{equation*}
  The other pairs can be easily checked the same way.
\end{proof}

Finite Lorentz transformations can be constructed by exponentiating linear combinations of these generators:
\begin{equation*}
  \Lambda = \exp\left(\sum_i \theta_i J_i + \sum_i \phi_i K_i \right),
\end{equation*}
where $\vec{\theta} = (\theta_1, \theta_2, \theta_3)$ are rotation parameters (angle and axis) and $\vec{\phi} = (\phi_1, \phi_2, \phi_3)$ are boost parameters (rapidity and direction).

\subsection{Kinematic Consequences}
\label{subsec:sr_kinematics}
The Lorentz transformations have profound implications for how different inertial observers measure time intervals and spatial lengths.
These effects are not mere illusions but are fundamental to the nature of spacetime.
% TODO: Kinematic consequences. These can be revisited later with a more geometric approach if desired.

\subsection{Structure of Flat Spacetime}
\label{subsec:sr_structure}
The theory of special relativity operates in a four-dimensional \emph{flat spacetime}.
Its geometric properties are defined by a metric tensor, $\eta_{\mu\nu}$, which is used to calculate spacetime intervals.
\begin{equation*}
  \eta_{\mu\nu} = \diag(-1,1,1,1).
\end{equation*}
The \emph{invariant spacetime interval}, $s^2$ (or often $ds^2$ for infinitesimal separations), between two events separated by $dx^\mu = (dx^0, dx^1, dx^2, dx^3)$ is given by:
\begin{equation*}
  ds^2 = \eta_{\mu\nu} dx^\mu dx^\nu = -(dx^0)^2 + (dx^1)^2 + (dx^2)^2 + (dx^3)^2.
\end{equation*}
This interval is invariant under Lorentz transformations (rotations and boosts).
The nature of the interval between two events determines their causal relationship:
\begin{itemize}
    \item \textbf{Timelike interval}: $ds^2 < 0$.
    It is possible for a massive particle to travel between these events; one event can causally affect the other.
    The proper time $d\tau$ is defined by $d\tau^2 = -ds^2/c^2$ (or $d\tau^2 = -ds^2$ if $c=1$).
    \item \textbf{Spacelike interval}: $ds^2 > 0$.
    It is not possible for a massive particle to travel between these events; they are causally disconnected in the sense that neither can lie in the past or future light cone of the other.
    The proper distance $d\sigma$ is $d\sigma^2 = ds^2$.
    \item \textbf{Null (or lightlike) interval}: $ds^2 = 0$.
    Only massless particles (like photons) can travel along such an interval; these define the light cones.
\end{itemize}

At every point in this spacetime, a \emph{tangent space} can be defined.
This is a vector space whose elements are 4-vectors, such as 4-displacement $dx^\mu$, 4-velocity $v^\mu$, 4-momentum $p^\mu$, etc.
These vectors represent physical quantities or directions in spacetime.
The metric tensor $\eta_{\mu\nu}$ defines an inner product on this tangent space: for two 4-vectors $A^\mu$ and $B^\nu$, their inner product is $\langle A, B \rangle = \eta_{\mu\nu} A^\mu B^\nu = A_\mu B^\mu = A^\mu B_\mu$.

A \emph{4-velocity} $v^\mu = \frac{dx^\mu}{d\tau}$ describes the rate of change of spacetime coordinates $x^\mu$ with respect to the proper time $\tau$ along a particle's worldline.
Proper time is the time measured by a clock moving with the particle.
For a massive particle, its 4-velocity is normalized such that:
\begin{equation*}
  \eta_{\mu\nu} v^\mu v^\nu = v^\mu v_\mu = -1 \quad (\text{with } c=1).
\end{equation*}
\begin{remark}
Throughout these notes, we use the $(-,+,+,+)$ metric signature convention, which leads to the normalization $v^\mu v_\mu = -1$ for timelike 4-velocities of massive particles.
\end{remark}
This normalization reflects that in the particle's own rest frame, its 4-velocity is purely in the time direction ($v^\mu_{rest} = (1,0,0,0)$ if $c=1$), and $d\tau^2 = -(dx^0_{rest})^2$.
The metric and the concept of the invariant interval are thus fundamental to understanding motion and causality in special relativity.

\section{Symmetries in Special Relativity: The Poincaré Group and Particle States}
\label{sec:poincare_particle_states}

This section explores the full symmetry group of flat spacetime, the Poincaré group, and its profound implications for defining fundamental particle properties.

\subsection{Introduction to the Poincaré Group}
\label{subsec:intro_poincare}
% TODO: Extension of the Lorentz group to include spacetime translations.
% TODO: The 10 generators of the Poincaré group (rotations, boosts, and 4D translations).
% TODO: User to fill in detailed explanations

\subsection{The Lie Algebra of the Poincaré Group}
\label{subsec:poincare_algebra}
% TODO: The commutation relations among the generators ($[P^\mu, P^\nu]$, $[M^{\mu\nu}, P^\rho]$, $[M^{\mu\nu}, M^{\rho\sigma}]$).
% TODO: User to fill in detailed explanations and derivations

\subsection{Noether's Theorem and Conservation Laws from Poincaré Invariance}
\label{subsec:noether_poincare}
% TODO: Brief statement of Noether's Theorem.
% TODO: Connection of Poincaré symmetries to conserved quantities:
% TODO:   - Energy (from time translation invariance).
% TODO:   - Momentum (from spatial translation invariance).
% TODO:   - Angular Momentum (from rotational invariance).
% TODO:   - Conservation related to boosts (motion of center of energy).
% TODO: User to fill in detailed explanations

\subsection{Casimir Invariants: Defining Intrinsic Particle Properties}
\label{subsec:casimir_invariants}
This subsection discusses operators that commute with all generators of the Poincaré algebra, leading to fundamental invariant properties of particles.

\subsubsection{What are Casimir Invariants?}
\label{subsubsec:what_are_casimirs}
% TODO: General definition of a Casimir invariant (or Casimir operator) for a Lie algebra.
% TODO: Their property of having constant eigenvalues on irreducible representations.
% TODO: User to fill in detailed explanations

\subsubsection{The $P^2$ Invariant: Mass}
\label{subsubsec:p_squared_invariant}
The first key Casimir invariant is $P^2 = P_\mu P^\mu$.

\paragraph{Proof that $P^2$ is a Casimir Invariant:}
% TODO: This involves showing that $[P^2, P^\sigma] = 0$ and $[P^2, M^{\rho\sigma}] = 0$.
\textit{[Proof to be filled in by user based on our previous discussion.]}

\paragraph{Physical Significance of $P^2$:}
The eigenvalue of $P^2$ is $m^2$ (or $-m^2 c^2$ depending on metric and $c$ conventions), where $m$ is the invariant rest mass of the particle.
This demonstrates that mass is a fundamental Poincaré-invariant characteristic.
% TODO: User to elaborate

\subsubsection{The $W^2$ Invariant: Spin}
\label{subsubsec:w_squared_invariant}
The second key Casimir invariant is $W^2 = W_\mu W^\mu$, where $W_\mu = \frac{1}{2} \epsilon_{\mu\nu\rho\sigma} M^{\nu\rho} P^\sigma$ is the Pauli-Lubanski pseudovector.
% TODO: Conceptual introduction to $W_\mu$ and $W^2$.
% TODO: Statement that $W^2$ is a Casimir invariant (proof can be noted as more complex and omitted or referenced).
% TODO: Its eigenvalue for massive particles: $-m^2 s(s+1)c^2$ (or similar depending on conventions), where $s$ is the spin.
% TODO: Its relation to helicity for massless particles.
% TODO: User to fill in detailed explanations

\subsection{The Role of Schur's Lemma: Why Casimir Eigenvalues Label Particles}
\label{subsec:schurs_lemma_role}
This subsection explains the mathematical theorem that underpins why Casimir invariants provide constant labels for irreducible representations.

\subsubsection{Statement of Schur's Lemma}
\label{subsubsec:schurs_lemma_statement}
% TODO: State the relevant version of Schur's Lemma:
% TODO: For a complex, finite-dimensional irreducible representation of a group, any operator that commutes
% TODO: with all representation operators must be a scalar multiple of the identity.
% TODO: User to provide a precise statement

\subsubsection{Proof of Schur's Lemma}
\label{subsubsec:schurs_lemma_proof}
\textit{[Proof to be filled in by user based on our previous discussion. This involves showing that the eigenspace of the commuting operator is an invariant subspace, then using irreducibility.]}

\subsubsection{Implication for Casimir Invariants}
\label{subsubsec:schur_implication_casimir}
Since Casimir invariants commute with all generators (and thus all elements of a representation), Schur's Lemma implies they act as a constant scalar (their eigenvalue) on any irreducible representation.
% TODO: User to elaborate on this connection

\subsection{Consequence: Classification of Particles in Special Relativity}
\label{subsec:particle_classification_sr}
% TODO: How the eigenvalues of $P^2$ (mass) and $W^2$ (spin) provide a fundamental classification scheme for elementary particles in relativistic quantum mechanics.
% TODO: Brief conceptual link to Wigner's classification of the irreducible representations of the Poincaré group.
% TODO: User to fill in detailed explanations


\section{Transition to Curved Spacetime (General Relativity)}
\label{sec:transition_gr}
The theory of special relativity, while revolutionary, is fundamentally a theory of flat spacetime.
It does not incorporate gravity.
Newtonian gravity, on the other hand, describes gravity as a force acting instantaneously at a distance, which is incompatible with the finite speed of light posited in special relativity.
General relativity emerges from the need to reconcile these ideas.
% TODO: Limitations of SR and Newtonian Gravity, The Equivalence Principle, Gravity as Spacetime Curvature

\subsection{The Principle of Equivalence}
% TODO: Placeholder

\subsection{Gravity as Spacetime Curvature}
% TODO: Placeholder

\section{The Mathematical Language of Curved Spacetime (Differential Geometry)}
\label{sec:math_gr}
To describe gravity as the curvature of spacetime, we need the tools of differential geometry.
Spacetime is no longer assumed to be flat but is instead a more general mathematical object called a manifold, equipped with a metric tensor that can vary from point to point.
% TODO: User can add subsections:
\subsection{Manifolds and Charts}
% TODO: Placeholder
\subsection{Tangent Spaces, Vectors, and Tensors in GR}
% TODO: Placeholder
\subsection{The Metric Tensor $g_{\mu\nu}$}
% TODO: Placeholder
\subsection{Covariant Derivative and Parallel Transport}
% TODO: Placeholder
\subsection{Geodesics: Paths of Freely Falling Particles}
% TODO: Placeholder
\subsection{Curvature: Riemann Tensor, Ricci Tensor, Ricci Scalar}
% TODO: Placeholder

\section{The Field Equations of Gravity: How Matter and Energy Dictate Spacetime Curvature}
\label{sec:efe}
% TODO: User can add content here
\subsection{The Stress-Energy Tensor $T_{\mu\nu}$}
% TODO: Placeholder
\subsection{The Einstein Tensor $G_{\mu\nu}$}
% TODO: Placeholder
\subsection{The Equations: $G_{\mu\nu} = \frac{8\pi G}{c^4} T_{\mu\nu}$}
% TODO: Placeholder

\section{Gravity of a Non-Rotating, Uncharged Spherical Mass (and its Observational Tests)}
\label{sec:schwarzschild_tests}
One of the first and most important solutions to Einstein's field equations describes the spacetime outside a static, spherically symmetric, uncharged massive object.
This is known as the Schwarzschild solution.

\subsection{The Metric for a Spherical Mass (Schwarzschild Metric)}
% TODO: Placeholder for derivation/presentation

\subsection{Properties: Event Horizon, Gravitational Singularity}
% TODO: Placeholder

\subsection{Application 1: Perihelion Precession of Orbits (e.g., Mercury)}
\label{subsec:mercury_precession}
The perihelion precession of Mercury’s orbit is a seminal observational phenomenon that challenged Newtonian gravity and provided empirical support for the relativistic description of gravitation.
Classical Newtonian mechanics predicts orbital motion based on an inverse-square central force, resulting in closed elliptical orbits.
However, observations show an additional precession unaccounted for by classical perturbations from other planets.
General relativity explains this anomaly through the curvature of spacetime caused by the Sun’s mass.
By modeling Mercury’s trajectory as a geodesic in the Schwarzschild metric, one derives corrections to the Newtonian orbital elements, resulting in a predicted advance of the perihelion that matches observations to high precision.
This section presents the derivation of the relativistic corrections to Mercury’s orbit, starting from the Schwarzschild solution to Einstein’s field equations and applying perturbation techniques to the geodesic equations of motion.

\subsection{Application 2: Relativistic Effects on Clocks and Timekeeping (e.g., for Satellites/GPS)}
% TODO: Placeholder for Gravitational Time Dilation, Combined Effects for GPS

\subsection{Application 3: The Bending of Light and Gravitational Lensing}
% TODO: Placeholder for Deflection of Light
\subsubsection{The "Einstein Cross" and other lensing phenomena}
% TODO: Placeholder for how multiple images form

\section{Beyond Simple Cases: Rotating and Charged Massive Objects}
\label{sec:kerr_etc}

\subsection{Need for More General Solutions}
% TODO: Placeholder

\subsection{Rotating Massive Objects (e.g., Kerr solution)}
% TODO: Placeholder for Frame-Dragging
\subsubsection{The Ergosphere}
% TODO: Placeholder for definition, boundaries, Penrose process concept

\subsection{Charged, Non-Rotating Massive Objects (e.g., Reissner-Nordström solution)}
% TODO: Placeholder

\subsection{Charged AND Rotating Massive Objects (e.g., Kerr-Newman solution)}
% TODO: Placeholder

\section{Further Striking Phenomena and Future Directions}
\label{sec:future_phenomena}
% TODO: Placeholder for Gravitational Waves, Black Hole Thermodynamics, Wormholes, Cosmology, Recap.

\end{document}
